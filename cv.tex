\documentclass[11pt,a4paper,sans]{moderncv}        % possible options include font size ('10pt', '11pt' and '12pt'), paper size ('a4paper', 'letterpaper', 'a5paper', 'legalpaper', 'executivepaper' and 'landscape') and font family ('sans' and 'roman')

% moderncv themes
\moderncvstyle{classic}                             % style options are 'casual' (default), 'classic', 'banking', 'oldstyle' and 'fancy'
\moderncvcolor{blue}                               % color options 'black', 'blue' (default), 'burgundy', 'green', 'grey', 'orange', 'purple' and 'red'
%\renewcommand{\familydefault}{\sfdefault}         % to set the default font; use '\sfdefault' for the default sans serif font, '\rmdefault' for the default roman one, or any tex font name
%\nopagenumbers{}                                  % uncomment to suppress automatic page numbering for CVs longer than one page

% character encoding
\usepackage[utf8]{inputenc}                       % if you are not using xelatex ou lualatex, replace by the encoding you are using
%\usepackage{CJKutf8}                              % if you need to use CJK to typeset your resume in Chinese, Japanese or Korean

\usepackage{url}
\usepackage{hyperref}

% adjust the page margins
\usepackage[scale=0.90]{geometry}
%\setlength{\hintscolumnwidth}{3cm}                % if you want to change the width of the column with the dates
%\setlength{\makecvtitlenamewidth}{10cm}           % for the 'classic' style, if you want to force the width allocated to your name and avoid line breaks. be careful though, the length is normally calculated to avoid any overlap with your personal info; use this at your own typographical risks...

% personal data
\name{}{Ranit Pradhan}
\address{West Bengal, India}% optional, remove / comment the line if not wanted; the "postcode city" and "country" arguments can be omitted or provided empty
\phone[mobile]{+91 9382615195}                   % optional, remove / comment the line if not wanted; the optional "type" of the phone can be "mobile" (default), "fixed" or "fax"
\email{pradhanranit0019@gmail.com}                               % optional, remove / comment the line if not wanted
\homepage{RanitPradhan.github.io}                         % optional, remove / comment the line if not wanted
\social[linkedin]{ranit-pradhan}                        % optional, remove / comment the line if not wanted

% bibliography adjustements (only useful if you make citations in your resume, or print a list of publications using BibTeX)
%   to show numerical labels in the bibliography (default is to show no labels)
\makeatletter\renewcommand*{\bibliographyitemlabel}{\@biblabel{\arabic{enumiv}}}\makeatother
%   to redefine the bibliography heading string ("Publications")
%\renewcommand{\refname}{Articles}

% bibliography with mutiple entries
%\usepackage{multibib}
%\newcites{book,misc}{{Books},{Others}}
%----------------------------------------------------------------------------------
%            content
%----------------------------------------------------------------------------------
\begin{document}
%\begin{CJK*}{UTF8}{gbsn}                          % to typeset your resume in Chinese using CJK
%-----       resume       ---------------------------------------------------------
\makecvtitle

\section{Summary}
\cvitem{}{Looking for a position in the field of Embedded Systems and IoT development where I can utilize my skills to work towards personal and professional development and contribute towards the prosperity of the organization. I am highly motivated and eager to learn new things.}

\section{Education}
\cventry{2019-2023 Ongoing}{B.Tech in Electrical and Computer Engineering}{Amrita Vishwa Vidyapeetham}{Kollam, Kerala, India}{}
{\textit{CGPA: 7.8/10}}{}  % arguments 3 to 6 can be left empty
\cventry{2019}{Belda Gangadhar Academy}{}{Paschim Medinipur, W.B., India}{}
{\textit{Percentage: 78.93\%  }}{}  % arguments 3 to 6 can be left empty
\cventry{2017}{Jankapur High School(H.S)}{}{), Paschim Medinipur, W.B., India}{}
{\textit{Percentage: 86.14\%  }}{}  % arguments 3 to 6 can be left empty

\section{Experience}
\cventry{July 2023 to Present}{Associate IoT Security Researcher}{Payatu}{}{}
{\begin{itemize}
\item Wireless Security (BLE, Wi-Fi, ZigBee)
\item Exploring in SDR
\item Tools- ExpLIoT, Alpha AWUS036AC, ZigBee Auditor, GnuRadio.
\end{itemize}
\textbf{Page link:}
\url{https://payatu.com}}

\cventry{January 2023 to June 2023}{IoT Hardware Security Research Intern}{Payatu}{}{}
{\begin{itemize}
\item Hardware Protocols(UART,SPI,I2C,JTAG), Wireless Protocols(Bluetooth, Wi-Fi, ZigBee)
\item Tools- ExpLIoT Nano, Diva board, ZigBee Auditor.
\end{itemize}
\textbf{Page link:}
\url{https://payatu.com}}

\cventry{June 2022 to July 2022}{Summer Intern at CeNSE}{Indian Institute of Science}{}{}
{\begin{itemize}
\item Pressure sensor data acquisition and Visualization using IoT.
\item Tools- Raspberry Pi 2B, ADS115, MCP3553, InfluxDB, Grafana.
\end{itemize}
\textbf{Page link:}
\url{http://www.cense.iisc.ac.in}}

\cventry{June 2022 to July 2022}{IoT Internship}{Emertxe Information Technologies}{}{}
{\begin{itemize}
\item Foundational Skills – C \verb !\& Linux.
\item IoT Skills – IoT Architecture, IoT Cloud Platform, IoT Solution Integration 
\item Embedded Skills –  Micro-controller programming
\item Tools – Debuggers, Cross-compilers, Editors and many more 
\end{itemize}
\textbf{Page link:}
\url{https://www.emertxe.com}}

\cventry{December 2020 to December 2022}{Member at Team bi0s}{Amrita School of Engineering}{}{}
{\begin{itemize}
\item Participated in various CTFs like Mitre, CSAW, Defcon etc.
\item Currently working on Embedded Security and Linux systems.
\item Mentoring first and second year student members.
\end{itemize}
\textbf{Page link:}
\url{http://bi0shardware.com}}

\cventry{November 2019 to January 2021}{Member at IEEE, Kerala Section}{Amrita School of Engineering}{}{}
{\begin{itemize}
\item Participated in IEEE Hackathons, Conferences, Webinars.
\item Undergone a Machine Learning workshop sponsored by \textbf {Megara Robotics Pvt. Ltd}.
\end{itemize}
\textbf{Certificate link:}
\url{https://raw.githubusercontent.com/RanitPradhan/Certificates/main/Certificate_Me.jpg}}

% arguments 3 to 6 can be left empty

\pagebreak

\section{Projects}
\cventry{July 2022}{\href{https://github.com/RanitPradhan/CeNSE-IoT}{Real-Time Monitoring of CeNSE Developed MEMS Pressure Sensor}}{}{}{}
{Data acquisition and real-time monitoring of a differential pressure sensor, manufactured in MEMS Packaging Lab of IISc. 
}

\cventry{October 2021}{\href{https://drive.google.com/file/d/1D3U0FYlAGn_UKEoKV1kc6jSVG9tNdkMj/view?usp=sharing}{Vaccine Verification using RFID-based secure authentication.}}{}{}{}
{The aim of the project was to use RFID technology in the fight against Covid-19. By following the process we can verify one is fully vaccinated with a good health or not, besides it we also tried to include contactless temperature verification.
}

\cventry{July 2020}{\href{https://iopscience.iop.org/article/10.1149/1945-7111/abc7e8/pdf}{An Ultra-Portable Vis-NIR Spectrometer for Chemometric Applications}}{}{}{}
{On-site material inspection and quality analysis of food and agricultural produce, which require portable sensing systems. A mini spectrometer is used for the measurements and the spectra data is analyzed using machine learning.
}
\cventry{November 2021}{\href{https://github.com/RanitPradhan/Academics/tree/master/Microcontroller/RTES-Project}{Accident Alert in Mist}}{}{}{}
{\textbf{STM32F103C4} microcontroller application for accident avoidance of vehicles in foggy areas. Simulation platforms like Proteus, STMCubeMX, ARM-Keil are used.
}
\cventry{May 2021}{\href{https://github.com/RanitPradhan/Academics/tree/master/Microcontroller/Staircase_LED}{Staircase LED using PIC}}{}{}{}
{Simple staircase LED controlling using \textbf{PIC16F877a} microcontroller. Simulation platforms - MPLAB and  Proteus are used.
}
\cventry{September 2020}{\href{https://github.com/RanitPradhan/bi0s/tree/master/Projects/COVID-19_Alert_Distance}{COVID-19 Alert Distance}}{}{}{}
{This project is related to the recent pandemic situation of COVID-19. A replica model to alert human to keep safe distance from each other. 
}

\cventry{January 2020}{\href{https://github.com/RanitPradhan/bi0s/tree/master/Projects/ac_to_dc_converter}{AC to 12V DC Converter}}{}{}{}
{The primary objective of this project was to glow a 12V LED strip using AC to DC converter. 
}


%\cventry{July 2020}{\href{https://github.com/RanitPradhan/docs/blob/main/Group_5_OS_Report.pdf}{A survey on state-of-the-art light weight, low energy operating system and technologies for wearable devices}}{}{}{}
%{The primary objective of this project was to  describe  and  explain about the light weight operating system, to know the characteristics, advantages and disadvantages of light weight OS. We concentrated on various kinds of Light weight, Low energy operating  systems and  their  applications  that  are used in this era. 
%}


\section{Courses and Mooc}
\cventry{May 2020}{\href{https://coursera.org/share/a6fdb8db507c58f7ae65a0ba2fb95eee}{The Arduino Platform and C Programming}}{}{}{}
{Issuing Organization: \textit{University of California,Irvine.}
}

\cventry{April 2020}{\href{https://raw.githubusercontent.com/RanitPradhan/Certificates/main/For_CV/JSON_data.JPG}{Working with JSON Data}}{}{}{}
{Issuing Organization: \textit{Real Python}.
}
\cventry{August 2020}{\href{https://raw.githubusercontent.com/RanitPradhan/Certificates/main/For_CV/Data_Visualization.JPG}{Data Visualization with Python}}{}{}{}
{Issuing Organization: \textit{Real Python} .
}
\cventry{July 2021}{\href{https://www.udemy.com/certificate/UC-1d258d4e-9893-49b2-8f4f-c87f361bb1d8/}{The Complete Front-End Web Development Course.}}{}{}{}
{Issuing Organization: \textit{Udemy} .
}

\section{Volunteer}
\cventry{January 2015}{Science Exhibition Project-1}{}{}{}
{Contributed in a model explanation of an Automated railway alarming system if there is any fault on the train-line, on the  
Platinum Jubilee celebration of Jankapur High School(H.S)}
\cventry{January 2019}{Science Exhibition Project-2}{}{}{}
{Contributed in a model explanation of an Automated water level alarming system, on the Centenary celebration of Belda Gangadhar Academy}
\cventry{December 2019}{Crowd Control Volunteer}{}{}{}
{Volunteered for crowd control in our Chancellor, Mata Amritanandamayi Devi's Birthday celebration.}
\cventry{March 2022}{Organizing Holi Celebration}{}{}{}
{Core team member for the arrangements of Amrita University Holi Celebration 2022.}

\section{Internships and Workshops}
\cventry{October 2019}{Hacktoberfest 2019}{}{}{}
{Attended conference of Digital Ocean,an introduction to Git and GitHub. It is a two days workshop taken by amfoss student club every year.}

\cventry{October 2020}{Machine learning Internship, IEEE}{}{}{}
{An online internship based on Data Science and ML. {\newline}
\textbf {Certificate Link:} {\newline}
\url{https://raw.githubusercontent.com/RanitPradhan/Certificates/main/Certificate_Me.jpg }}{}

\section{Achievements}
\cventry{November 2021}{\href{https://api.accredible.com/v1/frontend/credential_website_embed_image/certificate/42894843}{CSAW'21 Embedded Security Challenge Finalist India}}{}{}{}
{Finalist for CSAW'21 ESC, India. Mostly challenges were based on Side Channel Attacks and Chipwhisperer Nano was used for the analysis.}

\cventry{October 2021}{\href{https://raw.githubusercontent.com/RanitPradhan/Certificates/main/IEEE_RFID_2021.png}{Runner-Up in IEEE RFID-TA 2021 Challenge}}{}{}{}
{Secured second place in this national ideathon with the topic \textbf{Vaccine Verification using RFID-based secure authentication}.
}

\cventry {September 2017}{\href{http://www.paschimbangavigyanmancha.org/}{Paschim Banga Vigyan Mancha}}{}{}{}
{Paschim Banga Vigyan Mancha award in 2009 and 2017 for getting 5th position in our district and 2nd position in my block respectively.}

\section{Coursework}

%\renewcommand{\listitemsymbol}{-~} % Changes the symbol used for lists

\cvitem{Core Courses}{IoT, Embedded Systems, Microcontrollers and Applications, Electric Machines, Digital Electronics, Microelectronic Circuits}{}
\cvitem{Lab Courses}{ Microcontroller and Architecture, Data Structure, Power Electronics, Python Object Oriented Programming, Digital Manufacturing.}{}

\section{Languages}
\cvitemwithcomment{English}{Full Professional Proficiency}{}
\cvitemwithcomment{Hindi}{Full Professional Proficiency}{}
\cvitemwithcomment{Bengali}{Full Professional Proficiency}{}
\cvitemwithcomment{Odia}{Full Communication Proficiency}{}

\section{Skills}
\cvitem{Languages}{Python, C, C++, SQL}
\cvitem{Core}{Embedded C, AVR, Networking, Robotics}
\cvitem{WebD}{HTML,CSS}
\cvitem{VCS}{Git, Jupyter Notebook}
\cvitem{Tools}{
  \begin{itemize}
    \item \textbf{Software} STM CubeMx, MPLAB, Arduino IDE, VS Code, MATLAB, LT Spice, Proteus, ARM-Keil, Logisim, Eagle CAD
    \item \textbf{Hardware} Arduino UNO, ESP(8266,32), Tiva C, RaspberryPi, Logic Analyzer, Sensors. 
  \end{itemize}
}
\cvitem{Soft Skills}{Team management, Leadership, Mentorship}

\section{Interests}
\cvitem{Technical}{IoT, Embedded Systems,Firmware, Robotics, Machine Learning, Contributing to Open Source}
\cvitem{Hobbies}{Travelling, Cricket}

\section{Personal Details}
\cvitem{DOB}{3rd August, 2000}
\cvitem{Address}{Amritapuri, Kollam, Kerala, India}
\cvitem{Status}{Student}

\clearpage
%\clearpage\end{CJK*}                              % if you are typesetting your resume in Chinese using CJK; the \clearpage is required for fancyhdr to work correctly with CJK, though it kills the page numbering by making \lastpage undefined
\end{document}


%% end of file `template.tex'.
